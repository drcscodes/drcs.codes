% Created 2022-01-31 Mon 22:33
% Intended LaTeX compiler: pdflatex
\documentclass[smaller]{beamer}
\usepackage[utf8]{inputenc}
\usepackage[T1]{fontenc}
\usepackage{graphicx}
\usepackage{grffile}
\usepackage{longtable}
\usepackage{wrapfig}
\usepackage{rotating}
\usepackage[normalem]{ulem}
\usepackage{amsmath}
\usepackage{textcomp}
\usepackage{amssymb}
\usepackage{capt-of}
\usepackage{hyperref}
\usepackage{verbatim, multicol, tabularx}
\usepackage{sourcecodepro}
\usepackage[T1]{fontenc}
\usepackage{amsmath,amsthm, amssymb, latexsym, listings, qtree}
\lstset{extendedchars=\true, inputencoding=utf8, frame=tb, aboveskip=1mm, belowskip=0mm, showstringspaces=false, columns=flexible, basicstyle={\footnotesize\ttfamily}, numbers=left, frame=single, breaklines=true, breakatwhitespace=true, tabsize=4,  keywordstyle=\color{blue}, identifierstyle=\color{violet}, stringstyle=\color{teal}, commentstyle=\color{darkgray}}
\setbeamertemplate{footline}[frame number]
\hypersetup{colorlinks=true,urlcolor=blue}
\usetheme{default}
\date{}
\title{Introduction to Go Programming}
\hypersetup{
 pdfauthor={},
 pdftitle={Introduction to Go Programming},
 pdfkeywords={},
 pdfsubject={},
 pdfcreator={Emacs 27.2 (Org mode 9.4.4)},
 pdflang={English}}
\begin{document}

\maketitle

\section{Introduction to Go Programming}
\label{sec:orgcb3594a}

\begin{frame}[label={sec:org09fa5af}]{The Go Programming Language}
Made by Google.
\end{frame}


\begin{frame}[label={sec:orgaf0f58b},fragile]{Installing Go}
 \begin{enumerate}
\item Download the Go distribution for your OS from \url{https://go.dev/dl/}
\item Install Go using the instructions for your OS at \url{https://go.dev/doc/install}
\end{enumerate}

If installation was successful, you should see this at the command line (\texttt{❯} is the command prompt.  The command to type appears after the prompt, and the output of the command is printed on the next line.):

\lstset{language=sh,label= ,caption= ,captionpos=b,numbers=none}
\begin{lstlisting}
❯ go version
go version go1.17.3 darwin/amd64
\end{lstlisting}
\end{frame}

\begin{frame}[label={sec:orgb8c1a61},fragile]{\texttt{GOPATH}}
 After installing Go, the \texttt{go} executable will be in your path.  But the \texttt{go} command (more later) can also

\begin{itemize}
\item instal executable utilities, and
\item store local copies of third party module dependencies to speed up builds.
\end{itemize}

These packages are stored in the file path named in the \texttt{GOPATH} environment variable.  If \texttt{GOPATH} is not set, default is \texttt{\$HOME/go}.   Good idea to set it explicitly by placing

\lstset{language=sh,label= ,caption= ,captionpos=b,numbers=none}
\begin{lstlisting}
export GOPATH=$HOME/go
export PATH=$PATH:$GOPATH/bin
\end{lstlisting}

in your \texttt{\$HOME/.profile} for bash, or \texttt{\$HOME/.zshenv} for zsh.  Alter the values above if you want to customize the \texttt{GOPATH} location.
\end{frame}

\begin{frame}[label={sec:orge0cb31f},fragile]{Hello, World!}
 Since the Kernighan and Ritchie classic, \emph{The C Programming Language}\footnote{\url{https://www.informit.com/store/c-programming-language-9780131103627}}, it's tradition to start with a program that simply print's \texttt{Hello, world} to the console.

Create a file named \texttt{hello.go} with the following contents:

\lstset{language=go,label=orgd95d491,caption= ,captionpos=b,numbers=none}
\begin{lstlisting}
package main

import "fmt"

func main()  {
	fmt.Println("Hello, world!")
}
\end{lstlisting}
\end{frame}

\begin{frame}[label={sec:org62d2fd5},fragile]{The \texttt{go} Command}
 go run
go build
\end{frame}
\end{document}
