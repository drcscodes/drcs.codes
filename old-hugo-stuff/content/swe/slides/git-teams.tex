\documentclass{beamer}

\newcommand{\course}{CS 2340 Objects and Design}
\newcommand{\lesson}{Git for Teams}
\newcommand{\code}{http://www.cc.gatech.edu/~simpkins/teaching/gatech/cs2340/code}

\author[Chris Simpkins] 
{Christopher Simpkins \\\texttt{chris.simpkins@gatech.edu}}
\institute[Georgia Tech] % (optional, but mostly needed)

\date[CS 1331]{}

\usepackage{verbatim, multicol, tabularx,}
\usepackage{amsmath,amsthm, amssymb, stmaryrd, latexsym, bm, listings, qtree}
\usepackage{framed}
\usepackage{xcolor}
\usepackage{media9}

% Use article math fonts instead of ugly sans-serif math font
\usefonttheme[onlymath]{serif}

\usepackage{tikz,pgfplots}
\let\oldquote=\quote
\let\endoldquote=\endquote
\colorlet{shadecolor}{cyan!15}
\renewenvironment{quote}{\begin{shaded*}\begin{oldquote}}{\end{oldquote}\end{shaded*}}


\DeclareMathOperator*{\argmin}{arg\!min}
\DeclareMathOperator*{\argmax}{arg\!max}

\lstset{
  extendedchars=\true,
  inputencoding=utf8,
  literate=
  {é}{{\'{e}}}1
  {è}{{\`{e}}}1
  {ê}{{\^{e}}}1
  {ë}{{\¨{e}}}1
  {û}{{\^{u}}}1
  {ù}{{\`{u}}}1
  {â}{{\^{a}}}1
  {à}{{\`{a}}}1
  {î}{{\^{i}}}1
  {ô}{{\^{o}}}1
  {ç}{{\c{c}}}1
  {Ç}{{\c{C}}}1
  {É}{{\'{E}}}1
  {Ê}{{\^{E}}}1
  {À}{{\`{A}}}1
  {Â}{{\^{A}}}1
  {Î}{{\^{I}}}1
  {Ö}{{\"O}}1
  {Ä}{{\"A}}1
  {Ü}{{\"U}}1
  {ö}{{\"o}}1
  {ä}{{\"a}}1
  {ü}{{\"u}}1
  {ß}{{\ss}}1
  ,
  aboveskip=1mm,
  belowskip=1mm,
  showstringspaces=false,
  columns=flexible,
  basicstyle={\scriptsize\ttfamily},
  numbers=left,
  frame=single,
  framextopmargin=0pt,
  framexbottommargin=0pt,
  breaklines=true,
  breakatwhitespace=true,
  keywordstyle=\color{blue},
  identifierstyle=\color{violet},
  stringstyle=\color{teal},
  commentstyle=\color{darkgray}
}

\setbeamertemplate{footline}[frame number]
\hypersetup{colorlinks=true,urlcolor=blue}
\logo{\includegraphics[height=.5cm]{../../images/ksu-letterhead-logo.png}}

\headheight = 0.05 in
\headsep = 0.05 in
\parskip = 0.05in
\parindent = 0.0in
\floatsep = 0.05in


% \beamerdefaultoverlayspecification{<+->}


\begin{document}

\begin{frame}
  \titlepage
\end{frame}




%------------------------------------------------------------------------
\begin{frame}[fragile]{Git for Teams}


\begin{itemize}
\item Tagging
\item Branching
\item Distributed Workflow
\end{itemize}


\end{frame}
%------------------------------------------------------------------------

%------------------------------------------------------------------------
\begin{frame}[fragile]{Tagging}


A tag is an alias for a commit.  Create an annotated tag with {\tt git tag -a}
\begin{lstlisting}[language=bash]
[chris@nijinsky ~/work/vcs/github/tomcat-todo]
$ git tag -a m1 -m "Milestone 1: initial demo to class."
\end{lstlisting}
This creates an annotated tag ({\tt -a}) called {\tt m1} with a tag message of  "Milestone 1: initial demo to class."  We can list tags:
\begin{lstlisting}[language=bash]
[chris@nijinsky ~/work/vcs/github/tomcat-todo]
$ git tag
m1
\end{lstlisting}
And, for annotated tags, we can show the tag's details:
\begin{lstlisting}[language=bash]
[chris@nijinsky ~/work/vcs/github/tomcat-todo]
$ git show m1
tag m1
Tagger: Chris Simpkins <chris.simpkins@gmail.com>
Date:   Wed Jun 12 07:14:17 2013 -0400

Milestone 1: initial demo to class.
...
\end{lstlisting}

\end{frame}
%------------------------------------------------------------------------

%------------------------------------------------------------------------
\begin{frame}[fragile]{Reverting to a Tag and Sharing Tags}


Lets say we add and commit {\tt src/main/webapp/stylesheets/main.css} and modify {\tt src/main/webapp/list.jsp} to use the stylesheet, but then we want to revert our working copy of the repo to the {\tt m1} tag.  We can do this with {\tt git checkout}
\vspace{-.05in}
\begin{lstlisting}[language=bash]
$ git checkout m1
\end{lstlisting}
Git gives us a message about being in a 'detached HEAD' state, and the new stylesheet changes are gone from our working copy.

We can get back to the current version of the rep by checking out master:
\begin{lstlisting}[language=bash]
$ git checkout master
\end{lstlisting}
\vspace{-.05in}
Tags aren't pushed to a remote unless you tell git to push the tags.  You can push all of your tags with {\tt git push --tags} or a specific tag with {\tt git push tag m1}.  You'll need to do this when you tag your milestones.

\end{frame}
%------------------------------------------------------------------------

%% %------------------------------------------------------------------------
%% \begin{frame}[fragile]{}


%% \begin{lstlisting}[language=bash]

%% \end{lstlisting}


%% \end{frame}
%% %------------------------------------------------------------------------


%% %------------------------------------------------------------------------
%% \begin{frame}[fragile]{Branching}


%% First: read Pro Git Chapter 4 to get the concepts, which we don't have time to cover in detail here.  We're just covering the basics you'll need to do your group project.\\

%% A branch is just a movable pointer to a commit.  Each repo has a {\tt master} branch by default.  Create a new branch with {\tt git branch}.  For example, say we have a task of 
%% \begin{lstlisting}[language=bash]

%% \end{lstlisting}


%% \end{frame}
%% %------------------------------------------------------------------------

%% %------------------------------------------------------------------------
%% \begin{frame}[fragile]{}


%% \begin{lstlisting}[language=bash]

%% \end{lstlisting}


%% \end{frame}
%% %------------------------------------------------------------------------

%% %------------------------------------------------------------------------
%% \begin{frame}[fragile]{}


%% \begin{lstlisting}[language=bash]

%% \end{lstlisting}


%% \end{frame}
%% %------------------------------------------------------------------------

%% %------------------------------------------------------------------------
%% \begin{frame}[fragile]{}


%% \begin{lstlisting}[language=bash]

%% \end{lstlisting}


%% \end{frame}
%% %------------------------------------------------------------------------

%% %------------------------------------------------------------------------
%% \begin{frame}[fragile]{}


%% \begin{lstlisting}[language=bash]

%% \end{lstlisting}


%% \end{frame}
%% %------------------------------------------------------------------------

% %------------------------------------------------------------------------
% \begin{frame}[fragile]{}


% \begin{lstlisting}[language=Java]

% \end{lstlisting}

% \begin{itemize}
% \item
% \end{itemize}


% \end{frame}
% %------------------------------------------------------------------------


\end{document}
