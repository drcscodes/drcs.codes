\documentclass[addpoints]{exam}

\usepackage{verbatim, multicol, tabularx,hyperref, tikz, enumitem}
\usepackage{amsmath,amsthm, amssymb, stmaryrd, latexsym, bm, listings, qtree}
\usepackage[margin=1in]{geometry}

\lstset{
  extendedchars=\true,
  inputencoding=utf8,
  literate=
  {é}{{\'{e}}}1
  {è}{{\`{e}}}1
  {ê}{{\^{e}}}1
  {ë}{{\¨{e}}}1
  {û}{{\^{u}}}1
  {ù}{{\`{u}}}1
  {â}{{\^{a}}}1
  {à}{{\`{a}}}1
  {î}{{\^{i}}}1
  {ô}{{\^{o}}}1
  {ç}{{\c{c}}}1
  {Ç}{{\c{C}}}1
  {É}{{\'{E}}}1
  {Ê}{{\^{E}}}1
  {À}{{\`{A}}}1
  {Â}{{\^{A}}}1
  {Î}{{\^{I}}}1
  {Ö}{{\"O}}1
  {Ä}{{\"A}}1
  {Ü}{{\"U}}1
  {ö}{{\"o}}1
  {ä}{{\"a}}1
  {ü}{{\"u}}1
  {ß}{{\ss}}1
  ,
  aboveskip=1mm,
  belowskip=1mm,
  showstringspaces=false,
  columns=flexible,
  basicstyle={\scriptsize\ttfamily},
  numbers=left,
  frame=single,
  framextopmargin=0pt,
  framexbottommargin=0pt,
  breaklines=true,
  breakatwhitespace=true,
  keywordstyle=\color{blue},
  identifierstyle=\color{violet},
  stringstyle=\color{teal},
  commentstyle=\color{darkgray}
}

\hypersetup{colorlinks=true,urlcolor=blue}

\headheight = 0.05 in
\headsep = 0.05 in
\parskip = 0.05in
\parindent = 0.0in
\floatsep = 0.05in

\DeclareMathOperator*{\argmin}{arg\!min}
\DeclareMathOperator*{\argmax}{arg\!max}

\title{Knowledge-Based AI Review}
\author{Artificial Intelligence}
\date{}

\begin{document}
\maketitle

\begin{questions}


\setcounter{section}{0} % So that next \section is 1

\question In the context of propositional logic, define sentence.

\begin{solution}[.5in]
An assertion about the world expressed in a **knowledge represeantation language**, like propositional logic.
\end{solution}

\question In the context of propositional logic, define knowledge base.

\begin{solution}[.5in]
A set of sentences.
\end{solution}

\question In the context of propositional logic, define axiom.

\begin{solution}[.5in]
Sentences taken as given -- not derived from other sentences, assumptions.
\end{solution}

\question In the context of propositional logic, define inference.

\begin{solution}[.5in]
Deriving new sentences from old sentences.
\end{solution}

\question Define grounding.

\begin{solution}[.5in]
Connection between logical reasoning and the real environment.  How do we know that KB is true in the real world.
\end{solution}

\question Define entailment.

\begin{solution}[1in]
\end{solution}

\question In the context of logical reasoning, define model.

\begin{solution}[1in]
\end{solution}

\question In the context of logical reasoning, define model checking.

\begin{solution}[1in]
\end{solution}

\question List two methods of establishing the truth of a logical sentence.

\begin{solution}[1in]
\end{solution}

\question Define satisfiability.

\begin{solution}[1in]
A sentence is **satisfiable** if it is true in, or satisfied by, *some* model
\end{solution}

\question How does logical theorem proving work?

\begin{solution}[1in]
\end{solution}

\question Translate the following English sentence into first-order logic: ``All that glitter is not gold.''

\begin{solution}[1in]
\end{solution}

\question Translate the following English sentence into first-order logic: ``There's someone for everyone.''

\begin{solution}[1in]
\end{solution}

\question Express the following Engilsh sentence in a formal ontological knowledge representation language: ``A hot dog is a sandwich.''

\begin{solution}[1in]
\end{solution}

\end{questions}

\end{document}
