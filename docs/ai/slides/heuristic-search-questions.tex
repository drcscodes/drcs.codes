\documentclass[]{exam}

\usepackage{verbatim, multicol, tabularx,hyperref, tikz, enumitem}
\usepackage{amsmath,amsthm, amssymb, stmaryrd, latexsym, bm, listings, qtree}
\usepackage[margin=1in]{geometry}

\lstset{
  extendedchars=\true,
  inputencoding=utf8,
  literate=
  {é}{{\'{e}}}1
  {è}{{\`{e}}}1
  {ê}{{\^{e}}}1
  {ë}{{\¨{e}}}1
  {û}{{\^{u}}}1
  {ù}{{\`{u}}}1
  {â}{{\^{a}}}1
  {à}{{\`{a}}}1
  {î}{{\^{i}}}1
  {ô}{{\^{o}}}1
  {ç}{{\c{c}}}1
  {Ç}{{\c{C}}}1
  {É}{{\'{E}}}1
  {Ê}{{\^{E}}}1
  {À}{{\`{A}}}1
  {Â}{{\^{A}}}1
  {Î}{{\^{I}}}1
  {Ö}{{\"O}}1
  {Ä}{{\"A}}1
  {Ü}{{\"U}}1
  {ö}{{\"o}}1
  {ä}{{\"a}}1
  {ü}{{\"u}}1
  {ß}{{\ss}}1
  ,
  aboveskip=1mm,
  belowskip=1mm,
  showstringspaces=false,
  columns=flexible,
  basicstyle={\scriptsize\ttfamily},
  numbers=left,
  frame=single,
  framextopmargin=0pt,
  framexbottommargin=0pt,
  breaklines=true,
  breakatwhitespace=true,
  keywordstyle=\color{blue},
  identifierstyle=\color{violet},
  stringstyle=\color{teal},
  commentstyle=\color{darkgray}
}

\hypersetup{colorlinks=true,urlcolor=blue}

\headheight = 0.05 in
\headsep = 0.05 in
\parskip = 0.05in
\parindent = 0.0in
\floatsep = 0.05in

\DeclareMathOperator*{\argmin}{arg\!min}
\DeclareMathOperator*{\argmax}{arg\!max}

\title{Heuristic Search Review}
\author{Artificial Intelligence}
\date{}

\begin{document}
\maketitle

\begin{questions}


\setcounter{section}{0} % So that next \section is 1

\question What is a heuristic function?

\begin{solution}[1in]
A heuristic function, $h(n)$, gives an estimated cost of the cheapest path from the state at node $n$ to a goal state.
\end{solution}

\question How do you use Best-First Search to implement $A^*$ search?

\begin{solution}[1in]
Remember that Best-First Search uses a priority queue to choose the next node for consideration, where the ordering of nodes in the priority queue is determined by the value of a function $f(n)$.  If the path cost from the initial state to a node $n$ is given by $g(n)$, then $f(n) = g(n) + h(n)$ is an estimate of the complete path cost from the initial state to a goal state tghrough node $n$.  $A^*$ search uses $f(n) = g(n) + h(n)$ to order the nodes in the priority queue.
\end{solution}

\question In this diagram, nodes are marked with their $A^*$ heuristic values, $f(n) = g(n) + h(n)$.  Which node will be the next node expanded by $A^*$?

\begin{center}
\includegraphics[height=1.5in]{aima-fig-03_18-d-after-expanding-rimnicu-vilcea.pdf}
\end{center}

\begin{solution}[.25in]
Faragas, because it has the least $f(n)$ value of all the unexpanded nodes.
\end{solution}

\question Define admissible heuristic.

\begin{solution}[1in]
An admissible heuristic never overestimates the cost of the path from a node a goal
\end{solution}

\newpage

\question Define consistent heuristic.

\begin{solution}[1in]
A heuristic $h(n)$ is consistent if, for every node $n$ and every successor $n'$ of $n$ generated by an action $a$, we have $h(n) \le c(n,a,n') + h(n')$.

  \includegraphics[height=1in]{aima-fig-03_19-consistency-triangle-inequality.pdf}

\end{solution}

\question Which important property does an admissible heuristic give $A^*$?

\begin{solution}[1in]
With an admissible heuristic, $A^*$ is cost-optimal.
\end{solution}

\question How does using a consistent heuristc improve $A^*$ compared to using an admissible, but not consistent heuristic?

\begin{solution}[1in]
  With a consistent heuristic, the first time $A^*$ reaches a state it will be on an optimal path, so it never re-adds a state to the frontier, never changes an entry in {\tt reached}.
\end{solution}

\question Describe three approaches to designing a heuristic function.

\begin{solution}[2.5in]
\begin{itemize}
\item Relaxing the problem definition
\item Storing precomputed solution costs for subproblems in a pattern database
\item Defining landmarks
\item Learning from experience
\end{itemize}
\end{solution}

\question Use one of the three approaches from the previous question to design an admissible heuristic for the Three Pitchers problem.  In the Three Pitchers problem thre are three unmarked pitchers -- an 8 L pitcher full of water, an empty 5 L pitcher, and an empty 3 L pitcher -- and an agent must reallocate the water so that one of the pitchers contains exactly 4 L of water.

\begin{solution}[2in]

\end{solution}


\end{questions}

\end{document}
