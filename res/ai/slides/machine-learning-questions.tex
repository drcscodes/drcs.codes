\documentclass[]{exam}

\usepackage{verbatim, multicol, tabularx,hyperref, tikz, enumitem}
\usepackage{amsmath,amsthm, amssymb, stmaryrd, latexsym, bm, listings, qtree}
\usepackage[margin=1in]{geometry}

\lstset{
  extendedchars=\true,
  inputencoding=utf8,
  literate=
  {é}{{\'{e}}}1
  {è}{{\`{e}}}1
  {ê}{{\^{e}}}1
  {ë}{{\¨{e}}}1
  {û}{{\^{u}}}1
  {ù}{{\`{u}}}1
  {â}{{\^{a}}}1
  {à}{{\`{a}}}1
  {î}{{\^{i}}}1
  {ô}{{\^{o}}}1
  {ç}{{\c{c}}}1
  {Ç}{{\c{C}}}1
  {É}{{\'{E}}}1
  {Ê}{{\^{E}}}1
  {À}{{\`{A}}}1
  {Â}{{\^{A}}}1
  {Î}{{\^{I}}}1
  {Ö}{{\"O}}1
  {Ä}{{\"A}}1
  {Ü}{{\"U}}1
  {ö}{{\"o}}1
  {ä}{{\"a}}1
  {ü}{{\"u}}1
  {ß}{{\ss}}1
  ,
  aboveskip=1mm,
  belowskip=1mm,
  showstringspaces=false,
  columns=flexible,
  basicstyle={\scriptsize\ttfamily},
  numbers=left,
  frame=single,
  framextopmargin=0pt,
  framexbottommargin=0pt,
  breaklines=true,
  breakatwhitespace=true,
  keywordstyle=\color{blue},
  identifierstyle=\color{violet},
  stringstyle=\color{teal},
  commentstyle=\color{darkgray}
}

\hypersetup{colorlinks=true,urlcolor=blue}

\headheight = 0.05 in
\headsep = 0.05 in
\parskip = 0.05in
\parindent = 0.0in
\floatsep = 0.05in

\DeclareMathOperator*{\argmin}{arg\!min}
\DeclareMathOperator*{\argmax}{arg\!max}

\title{Machine Learning Review (AIMA 19.1-19.6)}
\author{Artificial Intelligence}
\date{}

\begin{document}
\maketitle

\begin{questions}


\setcounter{section}{0} % So that next \section is 1

According to Tom Mitchell, machine learning is the study of algorithms that

\begin{itemize}
\item improve their performance {\tt P}
\item at some task {\tt T}
\item with experience {\tt E}.
\end{itemize}

A well-defined learning task is given by {\tt <P, T, E>}.

Formulate the following problems according to Tom Mitchell's machine learning problem specification (see \href{https://drcs.codes/deep-learning/slides/machine-learning.pdf}{Machine Learning Slides}) and the specification our textbook. For each of the following problems specify:

\begin{itemize}
\item The task {\tt T},
\item The performance measure {\tt P},
\item The experience {\tt E},
\item The target function $f: \mathcal{X} \rightarrow \mathcal{Y}$, that is,
\begin{itemize}
\item the input space $\mathcal{X}$, and
\item the output space $\mathcal{Y}$.
\end{itemize}
\end{itemize}

Remember that a function maps a domain to a co-domain, and these domains are sets.

\question Medical diagnosis: A patient walks in with a medical history and some symptoms, and you want to identify the problem.

\ifprintanswers
\begin{solution}
\begin{itemize}
\item Task, {\tt T}: diagnose problem
\item Performance, {\tt P}: diagnosis is correct or incorrect
\item Experience, {\tt E}: $<medical-history, symptoms>$
\item Target function $f: \mathcal{X} \rightarrow \mathcal{Y}$:
\begin{itemize}
\item $\mathcal{X} = \{\vec{x} \vert x_1 \in \{family-history-heart-disease\}, x_2 \in \mathbb{R} = cholesterol-level \}$ and other such features
\item $\mathcal{Y} = \{disease_1, disease_2, ..., disease_n \}$
\end{itemize}
\end{itemize}
\end{solution}
\else
\vspace{3in}
\fi


\question

\begin{solution}[.5in]

\end{solution}

\question

\begin{solution}[.5in]

\end{solution}

\question

\begin{solution}[.5in]

\end{solution}


\end{questions}

\end{document}
