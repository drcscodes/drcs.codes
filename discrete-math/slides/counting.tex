% Created 2022-07-31 Sun 18:56
% Intended LaTeX compiler: pdflatex
\documentclass[smaller]{beamer}
\usepackage[utf8]{inputenc}
\usepackage[T1]{fontenc}
\usepackage{graphicx}
\usepackage{longtable}
\usepackage{wrapfig}
\usepackage{rotating}
\usepackage[normalem]{ulem}
\usepackage{amsmath}
\usepackage{amssymb}
\usepackage{capt-of}
\usepackage{hyperref}
\usepackage{verbatim, multicol, tabularx}
\usepackage{sourcecodepro}
\usepackage[T1]{fontenc}
\usepackage{amsmath,amsthm, amssymb, latexsym, listings, qtree}
\lstset{extendedchars=\true, inputencoding=utf8, frame=tb, aboveskip=1mm, belowskip=0mm, showstringspaces=false, columns=flexible, basicstyle={\footnotesize\ttfamily}, numbers=left, frame=single, breaklines=true, breakatwhitespace=true, tabsize=4,  keywordstyle=\color{blue}, identifierstyle=\color{violet}, stringstyle=\color{teal}, commentstyle=\color{darkgray}}
\setbeamertemplate{footline}[frame number]
\hypersetup{colorlinks=true,urlcolor=blue}
\usetheme{default}
\date{}
\title{Counting}
\hypersetup{
 pdfauthor={},
 pdftitle={Counting},
 pdfkeywords={},
 pdfsubject={},
 pdfcreator={Emacs 28.1 (Org mode 9.5.2)},
 pdflang={English}}
\begin{document}

\maketitle

\section{Counting}
\label{sec:orgf37822b}

\begin{frame}[label={sec:org40b299d}]{Counting}
The number of objects of interest.

At the end of this lesson you will

\begin{itemize}
\item understand the basic collections into which we can group objects, and
\item know how to count several kinds of subsets of objects within collecions.
\end{itemize}
\end{frame}

\begin{frame}[label={sec:org7196737}]{Lists}
A list is an ordered sequence of elements denoted in literal form with objects -- the elements or entries - enclosed in parentheses and separated by commas, e.g.:

$$
(a, b, c)
$$

Unlike sets, repetition is allowed and order matters:

\begin{eqnarray*}
(a, b, a)   & \ne & (a, b, b, a)\\
(a, b, c)   & \ne & (c, b, a)
\end{eqnarray*}

Remember that in sets "repeated" elements represent a single element and the order in which the elements are enumerated is not significant:

\begin{eqnarray*}
\{a, b, a\} & = & \{a, b, b, a\}\\
\{a, b, c\} & = & \{c, b, a\}
\end{eqnarray*}
\end{frame}

\begin{frame}[label={sec:org0555484},fragile]{Aside: Lists and Tuples in Programming}
 Programming languages typically distinguish between \alert{lists}, which are mutable (elements can be changed), and \alert{tuples}, which are immutable.  The mathematical notion of a list we discuss here corresponds to the programming notion of a tuple, and the notation often mirrors the mathematical notation with the same semantics.

\lstset{language=Python,label= ,caption= ,captionpos=b,numbers=none}
\begin{lstlisting}
>>> ['a', 'b', 'c'] # A Python list
>>> ('a', 'b', 'c') # A Python tuple
>>> {'a', 'b', 'c'} # A Python set
\end{lstlisting}
\end{frame}

\begin{frame}[label={sec:org3fc185e}]{Multiplication Principle}
The number of possible lists of length \(n\) with \(a_1\) possible choices for the first element, \(a_2\) choices for the second element, and so on is the product

$$
\prod_{i = 1}^{n} a_i
$$

\begin{block}{Example:}
If the set of vowels is \(V = \{a, e, i, o, u\}\), the set of consonants, \(C\), is the set of the remaining 21 letters in the 26-letter English alphabet, and every 3-letter word must start and end with a consonant with a vowel in the middle, then \(|C| = 21\), \(|V| = 5\) and the number of 3-letter words is \(21 \cdot 5 \cdot 21 = 2205\).
\end{block}
\end{frame}

\begin{frame}[label={sec:orgd193c68}]{Addition Principle}
If a finite set X can be decomposed as a union \(X = X_1 \cup X_2 \cup \dots \cup X_n\), where \(Xi \cap Xj = \emptyset\) whenever \(i \ne j\), then \(|X| = |X_1| + |X_2| + \dots + |X_n|\).
\end{frame}

\begin{frame}[label={sec:org2ba5901}]{Subtraction Principle}
If \(X\) is a subset of a finite set \(U\), then \(|overline{X}| = |U| - |X|\).
\end{frame}

\begin{frame}[label={sec:orgf94f3c1}]{Factorials}
If \(n\) is non-negative, then "n factorial", denoted \(n!\) is

\begin{equation*}
      n! = \begin{cases}
               1                 & \text{if } n \le 1\\
               n * (n-1)!        & \text{otherwise}
           \end{cases}
\end{equation*}

Put another way, \(0! = 1\), \(1! = 1\), and for \(n > 1\), \(n! = n \dot (n - 1) \dot (n - 2) \cdots 1\)
\end{frame}

\begin{frame}[label={sec:org46a199e}]{Permutations}
A permutation of a set \(X\) is a non-repetitive list of the elements of \(X\).  If \(|X| = n\), then the number of permutations of \(X\) is \(n!\).
\end{frame}

\begin{frame}[label={sec:org701f159}]{k-Permutations}
A k-permutation of a set \(X\) is a non-repetitive k-element list of elements from \(X\).  If \(|X| = n\), then the number of k-permutations of \(X\) is

$$
P(n, k) = \frac{n!}{(n - k)!}
$$

If \(k > n\), then \(P(n, k) = 0\).
\end{frame}

\begin{frame}[label={sec:org411ad7f}]{Combination}
A k-combination is a k-element subset of a set.  If a set has \(n\) elements, then the number of k-combinations is

$$
\binom{n}{k} = \frac{n!}{k!(n - k)!}
$$

\(\binom{n}{k}\) is sometimes written \(C(n, k)\).
\end{frame}
\end{document}