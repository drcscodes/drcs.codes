% Created 2022-08-17 Wed 12:44
% Intended LaTeX compiler: pdflatex
\documentclass[smaller, aspectratio=1610]{beamer}
\usepackage[utf8]{inputenc}
\usepackage[T1]{fontenc}
\usepackage{graphicx}
\usepackage{longtable}
\usepackage{wrapfig}
\usepackage{rotating}
\usepackage[normalem]{ulem}
\usepackage{amsmath}
\usepackage{amssymb}
\usepackage{capt-of}
\usepackage{hyperref}
\setbeamertemplate{navigation symbols}{}
\usepackage{verbatim, multicol, tabularx}
\usepackage{sourcecodepro}
\usepackage[T1]{fontenc}
\usepackage{amsmath,amsthm, amssymb, latexsym, listings, qtree}
\lstset{extendedchars=\true, inputencoding=utf8, frame=tb, aboveskip=1mm, belowskip=0mm, showstringspaces=false, columns=fixed, basicstyle={\footnotesize\ttfamily}, numbers=left, frame=single, breaklines=true, breakatwhitespace=true, tabsize=4,  keywordstyle=\color{blue}, identifierstyle=\color{violet}, stringstyle=\color{teal}, commentstyle=\color{darkgray}, upquote=false, literate={'}{\textquotesingle}1}
\setbeamertemplate{footline}[frame number]
\hypersetup{colorlinks=true,urlcolor=blue,bookmarks=true}
\setlength{\parskip}{.25\baselineskip}
\usetheme{default}
\date{}
\title{Strings}
\hypersetup{
 pdfauthor={},
 pdftitle={Strings},
 pdfkeywords={},
 pdfsubject={},
 pdfcreator={Emacs 28.1 (Org mode 9.5.2)},
 pdflang={English}}
\begin{document}

\maketitle


\section{Strings}
\label{sec:orgff12cda}

\begin{frame}[label={sec:org44313ee},fragile]{Strings}
 Three ways to define string literals:

\begin{itemize}
\item with single quotes: 'Ni!'

\item double quotes: "Ni!"

\item Or with triples of either single or double quotes, which creates a multi-line string:

\lstset{language=Python,label= ,caption= ,captionpos=b,numbers=none}
\begin{lstlisting}
    >>> """I do HTML for them all,
    ... even made a home page for my dog."""
    'I do HTML for them all,\neven made a home page for my dog.'
\end{lstlisting}
\end{itemize}
\end{frame}

\begin{frame}[label={sec:orgb27290c},fragile]{Strings}
 Note that the REPL echoes the value with a \texttt{\textbackslash{}n} to represent the newline character. Use the print function to get your intended output:

\lstset{language=Python,label= ,caption= ,captionpos=b,numbers=none}
\begin{lstlisting}
>>> nerdy = """I do HTML for them all,
... even made a home page for my dog."""
>>> nerdy
'I do HTML for them all,\neven made a home page for my dog.'
>>> print(nerdy)
I do HTML for them all,
even made a home page for my dog.
\end{lstlisting}

That's pretty \href{http://bravehunde.org}{nerdy}.
\end{frame}

\begin{frame}[label={sec:org4dd8b3a},fragile]{Strings}
 Choice of quote character is usually a matter of taste, but the choice can sometimes buy convenience. If your string contains a quote character you can either escape it:

\lstset{language=Python,label= ,caption= ,captionpos=b,numbers=none}
\begin{lstlisting}
>>> journey = 'Don\'t stop believing.'
\end{lstlisting}

or use the other quote character:

\lstset{language=Python,label= ,caption= ,captionpos=b,numbers=none}
\begin{lstlisting}
>>> journey = "Don't stop believing."
\end{lstlisting}

\begin{block}{Active Review}
\begin{itemize}
\item How does Python represent the value of the variable \texttt{journey}, that is, how is it echoed by the REPL?
\end{itemize}
\end{block}
\end{frame}

\begin{frame}[label={sec:org55b951b},fragile]{String Operations}
 Because strings are sequences we can get a string's length with \texttt{len()}:

\lstset{language=Python,label= ,caption= ,captionpos=b,numbers=none}
\begin{lstlisting}
>>> i = "team"
>>> len(i)
4
\end{lstlisting}

and access characters in the string by index (offset from beginning – first index is 0) using \texttt{[]}:

\lstset{language=Python,label= ,caption= ,captionpos=b,numbers=none}
\begin{lstlisting}
>>> i[1]
'e'
\end{lstlisting}

Note that the result of an index access is a string:

\lstset{language=Python,label= ,caption= ,captionpos=b,numbers=none}
\begin{lstlisting}
>>> type(i[1])
<class 'str'>
>>> i[3] + i[1]
'me'
>>> i[-1] + i[1] # Note that a negative index goes from the end
'me'
\end{lstlisting}

\begin{block}{Active Review}
\begin{itemize}
\item What is the index of the first character of a string?
\item What is the index of the last character of a string?
\end{itemize}
\end{block}
\end{frame}

\begin{frame}[label={sec:org29f4c31},fragile]{String Slicing}
 \texttt{[:end]} gets the first characters up to but not including \texttt{end}

\lstset{language=Python,label= ,caption= ,captionpos=b,numbers=none}
\begin{lstlisting}
>>> al_gore = "manbearpig"
>>> al_gore[:3]
'man'
\end{lstlisting}

\texttt{[begin:end]} gets the characters from \texttt{begin} up to but not including end

\lstset{language=Python,label= ,caption= ,captionpos=b,numbers=none}
\begin{lstlisting}
>>> al_gore[3:7]
'bear'
\end{lstlisting}

\texttt{[begin:]} gets the characters from \texttt{begin} to the end of the string

\lstset{language=Python,label= ,caption= ,captionpos=b,numbers=none}
\begin{lstlisting}
>>> al_gore[7:]
'pig'
>>>
\end{lstlisting}

\begin{block}{Active Review}
\begin{itemize}
\item What is the relationship between the ending index of a slice and the beginning index of a slice beginning right after the first slice?
\end{itemize}
\end{block}
\end{frame}

\begin{frame}[label={sec:orgcfb24ca},fragile]{String Methods}
 \texttt{str} is a class (you'll learn about classes later) with many methods (a method is a function that is part of an object). Invoke a method on a string using the dot operator.

\texttt{str.find(substr)} returns the index of the first occurence of
\texttt{substr} in \texttt{str}

\lstset{language=Python,label= ,caption= ,captionpos=b,numbers=none}
\begin{lstlisting}
>>> 'foobar'.find('o')
1
\end{lstlisting}

\begin{block}{Active Review}
\begin{itemize}
\item Write a string slice expression that returns the username from an email address, e.g., for 'bob@aol.com' it returns 'bob'.
\item Write a string slice expression that returns the host name from an email address, e.g., for 'bob@aol.com' it returns 'aol.com'.
\end{itemize}
\end{block}
\end{frame}

\begin{frame}[label={sec:orgbc22626},fragile]{String Interpolation with \%}
 The old-style (2.X) string format operator, \%, takes a string with format
specifiers on the left, and a single value or tuple of values on the right,
and substitutes the values into the string according to the conversion
rules in the format specifiers. For example:

\lstset{language=Python,label= ,caption= ,captionpos=b,numbers=none}
\begin{lstlisting}
>>> "%d %s %s %s %f" % (6, 'Easy', 'Pieces', 'of', 3.14)
'6 Easy Pieces of 3.140000'
\end{lstlisting}

Here are the conversion rules:

\begin{itemize}
\item \%s string
\item \%d decimal integer
\item \%x hex integer
\item \%o octal integer
\item \%f decimal float
\item \%e exponential float
\item \%g decimal or exponential float
\item \%\% a literal
\end{itemize}
\end{frame}

\begin{frame}[label={sec:orga202cf7},fragile]{String Formatting with \texttt{\%}}
 Specify field widths with a number between \texttt{\%} and conversion rule:

\lstset{language=Python,label= ,caption= ,captionpos=b,numbers=none}
\begin{lstlisting}
>>> sunbowl2012 = [('Georgia Tech', 21), ('USC', 7)]
>>> for team in sunbowl2012:
...     print('%14s %2d' % team)
...
Georgia Tech 21
USC           7
\end{lstlisting}
Fields right-aligned by default. Left-align with - in front of field width:

\lstset{language=Python,label= ,caption= ,captionpos=b,numbers=none}
\begin{lstlisting}
>>> for team in sunbowl2012:
...     print('%-14s %2d' % team)
...
Georgia Tech 21
USC           7
\end{lstlisting}

Specify n significant digits for floats with a .n after the field width:

\lstset{language=Python,label= ,caption= ,captionpos=b,numbers=none}
\begin{lstlisting}
>>> '%5.2f' % math.pi
' 3.14'
\end{lstlisting}
Notice that the field width indludes the decimal point and output is
left-padded with spaces
\end{frame}

\begin{frame}[label={sec:org4e88f73},fragile]{String Interpolation with \texttt{str.format()}}
 Python 3.0 - 3.5 interpolation was done with the string method \texttt{format}:

\lstset{language=Python,label= ,caption= ,captionpos=b,numbers=none}
\begin{lstlisting}
>>> "{} {} {} {} {}".format(6, 'Easy', 'Pieces', 'of', 3.14)
'6 Easy Pieces of 3.14'
\end{lstlisting}

Old-style formats only resolve arguments by position. New-style
formats can take values from any position by putting the position
number in the \{\} (positions start with 0):

\lstset{language=Python,label= ,caption= ,captionpos=b,numbers=none}
\begin{lstlisting}
>>> "{4} {3} {2} {1} {0}".format(6, 'Easy', 'Pieces', 'of', 3.14)
'3.14 of Pieces Easy 6'
\end{lstlisting}

Can also use named arguments, like functions:

\lstset{language=Python,label= ,caption= ,captionpos=b,numbers=none}
\begin{lstlisting}
>>> "{count} pieces of {kind} pie".format(kind='punkin', count=3)
'3 pieces of punkin pie'
\end{lstlisting}

Or dictionaries (note that there's one dict argument, number 0):

\lstset{language=Python,label= ,caption= ,captionpos=b,numbers=none}
\begin{lstlisting}
>>> "{0[count]} pieces of {0[kind]} pie".format({'kind':'punkin',
'count':3})
'3 pieces of punkin pie'
\end{lstlisting}
\end{frame}

\begin{frame}[label={sec:org62fce8c},fragile]{String Formatting with \texttt{str.format()}}
 Conversion types appear after a colon:

\lstset{language=Python,label= ,caption= ,captionpos=b,numbers=none}
\begin{lstlisting}
>>> "{:d} {} {} {} {:f}".format(6, 'Easy', 'Pieces', 'of', 3.14)
'6 Easy Pieces of 3.140000'
\end{lstlisting}

Argument names can appear before the :, and field formatters appear
between the : and the conversion specifier (note the < and > for left
and right alignment):

\lstset{language=Python,label= ,caption= ,captionpos=b,numbers=none}
\begin{lstlisting}
>>> for team in sunbowl2012:
...     print('{:<14s} {:>2d}'.format(team[0], team[1]))
...
Georgia Tech 21
USC           7
\end{lstlisting}

You can also unpack the tuple to supply its elements as individual
arguments to format (or any function) by prepending tuple with *:

\lstset{language=Python,label= ,caption= ,captionpos=b,numbers=none}
\begin{lstlisting}
>>> for team in sunbowl2012:
...     print('{:<14s} {:>2d}'.format(*team))
...
Georgia Tech 21
USC           7
\end{lstlisting}
\end{frame}

\begin{frame}[label={sec:org6b25336},fragile]{f-Strings}
 Python 3.6 introduced a much more convenient inline string interpolator.  Prepend \texttt{f} to the opening quote, enclose arbitrary Python expressions in culy braces (\texttt{\{\}}), and put formatters similar to \texttt{str.format()} after colons.

\lstset{language=Python,label= ,caption= ,captionpos=b,numbers=none}
\begin{lstlisting}
>>> for team, score in sunbowl2012:       # Tuple-unpacking assignment
...     print(f'{team:<14s} {score:>2d}')
...
Georgia Tech   21
USC             7
\end{lstlisting}
\end{frame}

\begin{frame}[label={sec:org06f2709},fragile]{Conclusion}
 \begin{itemize}
\item Strings are a kind of \texttt{Sequence}
\item Unlike some other languages, it's not a \texttt{Sequence[char]} -- single characters are also \texttt{str} s
\item Strings are immutable, so operations that "modify" strings actually return new strings containing the modificaitons
\end{itemize}
\end{frame}
\end{document}