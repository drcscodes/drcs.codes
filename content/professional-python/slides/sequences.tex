% Created 2022-08-17 Wed 12:44
% Intended LaTeX compiler: pdflatex
\documentclass[smaller, aspectratio=1610]{beamer}
\usepackage[utf8]{inputenc}
\usepackage[T1]{fontenc}
\usepackage{graphicx}
\usepackage{longtable}
\usepackage{wrapfig}
\usepackage{rotating}
\usepackage[normalem]{ulem}
\usepackage{amsmath}
\usepackage{amssymb}
\usepackage{capt-of}
\usepackage{hyperref}
\setbeamertemplate{navigation symbols}{}
\usepackage{verbatim, multicol, tabularx}
\usepackage{sourcecodepro}
\usepackage[T1]{fontenc}
\usepackage{amsmath,amsthm, amssymb, latexsym, listings, qtree}
\lstset{extendedchars=\true, inputencoding=utf8, frame=tb, aboveskip=1mm, belowskip=0mm, showstringspaces=false, columns=fixed, basicstyle={\footnotesize\ttfamily}, numbers=left, frame=single, breaklines=true, breakatwhitespace=true, tabsize=4,  keywordstyle=\color{blue}, identifierstyle=\color{violet}, stringstyle=\color{teal}, commentstyle=\color{darkgray}, upquote=false, literate={'}{\textquotesingle}1}
\setbeamertemplate{footline}[frame number]
\hypersetup{colorlinks=true,urlcolor=blue,bookmarks=true}
\setlength{\parskip}{.25\baselineskip}
\usetheme{default}
\date{}
\title{Sequences}
\hypersetup{
 pdfauthor={},
 pdftitle={Sequences},
 pdfkeywords={},
 pdfsubject={},
 pdfcreator={Emacs 28.1 (Org mode 9.5.2)},
 pdflang={English}}
\begin{document}

\maketitle

\section{Data Structures}
\label{sec:orgbee7a9d}

\begin{frame}[label={sec:orgbe50946}]{Built-in Data Structures}
Values can be collected in data structures:

\begin{itemize}
\item Lists
\item Tuples
\item Dictionaries
\item Sets
\end{itemize}

See the \href{https://docs.python.org/3.6/library/stdtypes.html}{Python documentation} for
complete details.

By the end of this lesson you will

\begin{itemize}
\item know how to use lists and tuples
\end{itemize}
\end{frame}

\begin{frame}[label={sec:org269a248},fragile]{Lists}
 A list is an indexed sequence of Python objects.

\begin{itemize}
\item Create a list with square brackets
\end{itemize}

\lstset{language=Python,label= ,caption= ,captionpos=b,numbers=none}
\begin{lstlisting}
>>> boys = ['Stan', 'Kyle', 'Cartman', 'Kenny']
\end{lstlisting}

\begin{itemize}
\item Create an empty list with empty square brackets or \texttt{list()} function
\end{itemize}

\lstset{language=Python,label= ,caption= ,captionpos=b,numbers=none}
\begin{lstlisting}
>>> empty = []
>>> leer = list()
\end{lstlisting}
\end{frame}

\begin{frame}[label={sec:orgcb6f168},fragile]{Accessing List Elements}
 Individual list elements are accessed by index.

\begin{itemize}
\item First element at index 0
\end{itemize}

\lstset{language=Python,label= ,caption= ,captionpos=b,numbers=none}
\begin{lstlisting}
>>> boys = ['Stan', 'Kyle', 'Cartman', 'Kenny']
>>> boys[0]
'Stan'
\end{lstlisting}

\begin{itemize}
\item Negative indexes offset from the end of the list backwards
\end{itemize}

\lstset{language=Python,label= ,caption= ,captionpos=b,numbers=none}
\begin{lstlisting}
>>> boys[-1]
'Kenny'
\end{lstlisting}

\begin{itemize}
\item Lists are mutable, meaning you can add, delete, and modify elements
\end{itemize}

\lstset{language=Python,label= ,caption= ,captionpos=b,numbers=none}
\begin{lstlisting}
>>> boys[2] = 'Eric'
>>> boys
['Stan', 'Kyle', 'Eric', 'Kenny']
\end{lstlisting}
\end{frame}

\begin{frame}[label={sec:orgd3e3f99},fragile]{Lists are Heterogeneous}
 Normally you store elements of the same type in a list, but you can mix element types

\lstset{language=Python,label= ,caption= ,captionpos=b,numbers=none}
\begin{lstlisting}
>>> mixed = [1, 'Two', 3.14]
>>> type(mixed[0])
<class 'int'>
>>> type(mixed[1])
<class 'str'>
>>> type(mixed[2])
<class 'float'>
\end{lstlisting}

\begin{itemize}
\item What's the length of the second element of \texttt{mixed} ?
\end{itemize}
\end{frame}

\begin{frame}[label={sec:org3d3902e},fragile]{Creating Lists from Strings}
 \begin{itemize}
\item Create a list from a string with str's \texttt{split()} function:
\end{itemize}

\lstset{language=Python,label= ,caption= ,captionpos=b,numbers=none}
\begin{lstlisting}
>>> grades_line = "90, 85, 92, 100"
>>> grades_line.split()
['90,', '85,', '92,', '100']
\end{lstlisting}

\begin{itemize}
\item By default \texttt{split()} uses whitespace to delimit elements. To use a different delimiter, pass as argument to \texttt{split()}:
\end{itemize}

\lstset{language=Python,label= ,caption= ,captionpos=b,numbers=none}
\begin{lstlisting}
>>> grades_line.split(',')
['90', ' 85', ' 92', ' 100']
\end{lstlisting}

\begin{itemize}
\item The \texttt{list()} function converts any iterable object (like sequences) to a list. Remember that strings are sequences of characters:
\end{itemize}

\lstset{language=Python,label= ,caption= ,captionpos=b,numbers=none}
\begin{lstlisting}
>>> list('abcdefghijklmnopqrstuvwxyz')
['a', 'b', 'c', 'd', 'e', 'f', 'g', 'h', 'i', 'j', 'k', 'l', 'm',
'n', 'o', 'p', 'q', 'r', 's', 't', 'u', 'v', 'w', 'x', 'y',
'z']
\end{lstlisting}

\begin{itemize}
\item Use the \texttt{split()} method to separate an email address in to user name and host name.
\end{itemize}
\end{frame}

\begin{frame}[label={sec:orge786234},fragile]{List Operators}
 The \texttt{in} operator tests for list membership. Can be negated with not:

\lstset{language=Python,label= ,caption= ,captionpos=b,numbers=none}
\begin{lstlisting}
>>> boys
['Stan', 'Kyle', 'Cartman', 'Kenny']
>>> 'Kyle' in boys
True
>>> 'Kyle' not in boys
False
\end{lstlisting}

\begin{itemize}
\item The + operator concatenates two lists:
\end{itemize}

\lstset{language=Python,label= ,caption= ,captionpos=b,numbers=none}
\begin{lstlisting}
>>> girls = ['Wendy', 'Annie', 'Bebe', 'Heidi']
>>> kids = boys + girls
>>> kids
['Stan', 'Kyle', 'Cartman', 'Kenny', 'Wendy', 'Annie', 'Bebe', 'Heidi']
\end{lstlisting}

\begin{itemize}
\item The * operator repeats a list to produce a new list:
\end{itemize}

\lstset{language=Python,label= ,caption= ,captionpos=b,numbers=none}
\begin{lstlisting}
>>> ['Ni'] * 5
['Ni', 'Ni', 'Ni', 'Ni', 'Ni']
\end{lstlisting}
\end{frame}

\begin{frame}[label={sec:orgf1ad233},fragile]{Functions on Lists}
 Python provides several built-in functions that take list parameters.

\begin{itemize}
\item \texttt{len(xs)} returns the number of elements in the list \texttt{xs} (more generally, the sequence \texttt{xs})
\end{itemize}

\lstset{language=Python,label= ,caption= ,captionpos=b,numbers=none}
\begin{lstlisting}
>>> kids
['Stan', 'Kyle', 'Cartman', 'Kenny', 'Wendy', 'Annie', 'Bebe', 'Heidi']
>>> len(kids)
8
\end{lstlisting}

\begin{itemize}
\item \texttt{min(xs)} returns the least element of \texttt{xs}, \texttt{max(xs)} returns the greatest
\end{itemize}

\lstset{language=Python,label= ,caption= ,captionpos=b,numbers=none}
\begin{lstlisting}
>>> min([8, 6, 7, 5, 3, 0, 9])
0
>>> max([8, 6, 7, 5, 3, 0, 9])
9
\end{lstlisting}

\begin{itemize}
\item What is \texttt{min(kids)}?
\end{itemize}
\end{frame}

\begin{frame}[label={sec:org6ecaf5e},fragile]{The \texttt{del} Statement}
 The \texttt{del} statement deletes variables.

\begin{itemize}
\item Each element of a list is a variable whose name is formed by indexing into the list with square brackets.
\end{itemize}

\lstset{language=Python,label= ,caption= ,captionpos=b,numbers=none}
\begin{lstlisting}
>>> boys = ['Stan', 'Kyle', 'Cartman', 'Kenny']
>>> boys[3]
'Kenny'
\end{lstlisting}

\begin{itemize}
\item Like any variable, a list element can be deleted with \texttt{del}
\end{itemize}

\lstset{language=Python,label= ,caption= ,captionpos=b,numbers=none}
\begin{lstlisting}
>>> del boys[3]
>>> boys
['Stan', 'Kyle', 'Cartman'] # You killed Kenny!
\end{lstlisting}

\begin{itemize}
\item A list variable is a variable, so you can delete the whole list
\end{itemize}

\lstset{language=Python,label= ,caption= ,captionpos=b,numbers=none}
\begin{lstlisting}
>>> del boys
>>> boys
Traceback (most recent call last):
File "<stdin>", line 1, in <module>
NameError: name 'boys' is not defined
\end{lstlisting}
\end{frame}

\begin{frame}[label={sec:org0de03ea},fragile]{List Methods}
 Methods are invoked on an object (an instance of a class) by appending a dot, \texttt{.}, and the method name.

\begin{itemize}
\item \texttt{xs.count(x)}: number of occurences of \texttt{x} in the sequence \texttt{xs}
\end{itemize}

\lstset{language=Python,label= ,caption= ,captionpos=b,numbers=none}
\begin{lstlisting}
>>> surfin_bird = "Bird bird bird b-bird's the word".split()
>>> surfin_bird
['Bird', 'bird', 'bird', "b-bird's", 'the', 'word']
>>> surfin_bird.count('bird')
2
\end{lstlisting}

\begin{itemize}
\item \texttt{xs.append(x)} adds the single element \texttt{x} to the end of \texttt{xs}
\end{itemize}

\lstset{language=Python,label= ,caption= ,captionpos=b,numbers=none}
\begin{lstlisting}
>>> boys.append('Butters')
>>> boys
['Stan', 'Kyle', 'Cartman', 'Kenny', 'Butters']
s.extend(t) adds the elements of t to the end of s
>>> boys.extend(['Tweak', 'Jimmy'])
>>> boys
['Stan', 'Kyle', 'Cartman', 'Kenny', 'Butters', 'Tweak', 'Jimmy']
\end{lstlisting}
\end{frame}

\begin{frame}[label={sec:org1dbc545},fragile]{List Methods}
 \begin{itemize}
\item \texttt{xs.remove(x)} removes the first occurrence of \texttt{x} in \texttt{xs}, or raises a \texttt{ValueError} if \texttt{x} is not in \texttt{xs}
\end{itemize}

\lstset{language=Python,label= ,caption= ,captionpos=b,numbers=none}
\begin{lstlisting}
>>> boys.remove('Kenny')
>>> boys
['Stan', 'Kyle', 'Cartman', 'Butters', 'Tweak', 'Jimmy']
>>> boys.remove('Professor Chaos')
Traceback (most recent call last):
File "<stdin>", line 1, in <module>
ValueError: list.remove(x): x not in list
\end{lstlisting}

\begin{itemize}
\item \texttt{xs.pop()} removes and returns the last element of the list
\end{itemize}

\lstset{language=Python,label= ,caption= ,captionpos=b,numbers=none}
\begin{lstlisting}
>>> boys
['Stan', 'Kyle', 'Cartman', 'Butters', 'Tweak', 'Jimmy']
>>> boys.pop()
'Jimmy'
>>> boys
['Stan', 'Kyle', 'Cartman', 'Butters', 'Tweak']
\end{lstlisting}
\end{frame}

\begin{frame}[label={sec:org2ebb3a0},fragile]{Slicing}
 Slicing lists works just like slicing strings (they're both sequences)

\begin{itemize}
\item Take the first two elements:
\end{itemize}

\lstset{language=Python,label= ,caption= ,captionpos=b,numbers=none}
\begin{lstlisting}
>>> boys = ['Stan', 'Kyle', 'Cartman', 'Butters', 'Tweak']
>>> boys[0:2]
['Stan', 'Kyle']
\end{lstlisting}

\begin{itemize}
\item Take every second element, starting with the first:
\end{itemize}

\lstset{language=Python,label= ,caption= ,captionpos=b,numbers=none}
\begin{lstlisting}
>>> boys[::2]
['Stan', 'Cartman', 'Tweak']
>>> boys[0:5:2] # same as above
['Stan', 'Cartman', 'Tweak']
\end{lstlisting}

\begin{itemize}
\item Take the second from the end:
\end{itemize}

\lstset{language=Python,label= ,caption= ,captionpos=b,numbers=none}
\begin{lstlisting}
>>> boys[-2]
'Butters'
\end{lstlisting}

Note that slice operations return new lists.

\begin{itemize}
\item What's the value of \texttt{boys[-1:1]} ?
\item What's the value of \texttt{boys[-1:1:-1]} ?
\item What's the value of \texttt{boys[::-1]} ?
\end{itemize}
\end{frame}

\begin{frame}[label={sec:org6aac59b},fragile]{Aliases}
 Aliasing occurs when two or more variables reference the same object

\begin{itemize}
\item Assignment from a variable creates an alias
\end{itemize}

\lstset{language=Python,label= ,caption= ,captionpos=b,numbers=none}
\begin{lstlisting}
>>> brats = boys
>>> boys
['Stan', 'Kyle', 'Cartman', 'Butters', 'Tweak']
>>> brats
['Stan', 'Kyle', 'Cartman', 'Butters', 'Tweak']
\end{lstlisting}

Now boys and brats are aliases.

\begin{itemize}
\item Changes to one are reflected in the other, becuase they reference the same object
\end{itemize}

\lstset{language=Python,label= ,caption= ,captionpos=b,numbers=none}
\begin{lstlisting}
>>> brats.append('Timmy')
>>> brats
['Stan', 'Kyle', 'Cartman', 'Butters', 'Tweak', 'Timmy']
>>> boys
['Stan', 'Kyle', 'Cartman', 'Butters', 'Tweak', 'Timmy']
\end{lstlisting}
\end{frame}

\begin{frame}[label={sec:org2643aa8},fragile]{Copies}
 Operators create copies

\lstset{language=Python,label= ,caption= ,captionpos=b,numbers=none}
\begin{lstlisting}
>>> brats + ['Bebe', 'Wendy']
['Stan', 'Kyle', 'Cartman', 'Butters', 'Tweak', 'Timmy', 'Bebe',
'Wendy']
>>> brats
['Stan', 'Kyle', 'Cartman', 'Butters', 'Tweak', 'Timmy']
\end{lstlisting}

You have to reassign to the list to make an update:

\lstset{language=Python,label= ,caption= ,captionpos=b,numbers=none}
\begin{lstlisting}
>>> brats = brats + ['Bebe', 'Wendy'] # could also use shortcut +=
>>> brats
['Stan', 'Kyle', 'Cartman', 'Butters', 'Tweak', 'Timmy', 'Bebe',
'Wendy']
\end{lstlisting}

Notice that after the reassignment, \texttt{brats} is no longer an alias of \texttt{boys}

\lstset{language=Python,label= ,caption= ,captionpos=b,numbers=none}
\begin{lstlisting}
>>> boys
['Stan', 'Kyle', 'Cartman', 'Butters', 'Tweak', 'Timmy']
\end{lstlisting}
\end{frame}

\begin{frame}[label={sec:orgada319c},fragile]{Slicing Creates Copies (Usually)}
 \begin{itemize}
\item Slice on the right hand side of an assignment creates a copy:
\end{itemize}

\lstset{language=Python,label= ,caption= ,captionpos=b,numbers=none}
\begin{lstlisting}
>>> first_two = boys[:2]
>>> first_two
['Stan', 'Kyle']
>>> first_two[0] = 'Stan the man'
>>> first_two
['Stan the man', 'Kyle']
>>> boys
['Stan', 'Kyle', 'Cartman', 'Butters', 'Tweak', 'Timmy']
\end{lstlisting}

\begin{itemize}
\item Slices on the left hand side allow for flexible assignment
\end{itemize}

\lstset{language=Python,label= ,caption= ,captionpos=b,numbers=none}
\begin{lstlisting}
>>> boys[0:2] = ['Randy', 'Sharon', 'Gerald', 'Sheila']
>>> boys
['Randy', 'Sharon', 'Gerald', 'Sheila', 'Cartman', 'Butters',
'Tweak', 'Timmy']
\end{lstlisting}
\end{frame}

\begin{frame}[label={sec:org488c425},fragile]{A Few More List Operations}
 You can combine the elements of a list to form a string with \texttt{str}'s \texttt{join()} method.

\lstset{language=Python,label= ,caption= ,captionpos=b,numbers=none}
\begin{lstlisting}
>>> aretha = ['R', 'E', 'S', 'P', 'E', 'C', 'T']
>>> "-".join(aretha)
'R-E-S-P-E-C-T'
\end{lstlisting}

\texttt{sorted()} function returns a new list

\lstset{language=Python,label= ,caption= ,captionpos=b,numbers=none}
\begin{lstlisting}
>>> sorted(aretha)
['C', 'E', 'E', 'P', 'R', 'S', 'T']
>>> aretha # Notice original is unchanged
['R', 'E', 'S', 'P', 'E', 'C', 'T']
\end{lstlisting}

\texttt{sort()} method modifies the list it is invoked on

\lstset{language=Python,label= ,caption= ,captionpos=b,numbers=none}
\begin{lstlisting}
>>> aretha.sort()
>>> aretha
['C', 'E', 'E', 'P', 'R', 'S', 'T']
\end{lstlisting}
\end{frame}

\begin{frame}[label={sec:orgac455b3},fragile]{Example: Grades}
 Start with a list representing a line from a gradebook file

\lstset{language=Python,label= ,caption= ,captionpos=b,numbers=none}
\begin{lstlisting}
>>> grades_line = ['Chris', 100, 90, 95]
>>> grades_line
['Chris', 100, 90, 95]
\end{lstlisting}

Get the sublist containing just the grades by slicing

\lstset{language=Python,label= ,caption= ,captionpos=b,numbers=none}
\begin{lstlisting}
>>> grades = grades_line[1:]
>>> grades
[100, 90, 95]
\end{lstlisting}

Sum the grades using Python's built-in \texttt{sum()} function

\lstset{language=Python,label= ,caption= ,captionpos=b,numbers=none}
\begin{lstlisting}
>>> sum(grades)
285
\end{lstlisting}

\begin{itemize}
\item And get the average by dividing by the number of grades
\end{itemize}

\lstset{language=Python,label= ,caption= ,captionpos=b,numbers=none}
\begin{lstlisting}
>>> sum(grades) / len(grades)
95.0
\end{lstlisting}
\end{frame}

\begin{frame}[label={sec:orgdcf87a6},fragile]{Tuples}
 Tuples are like lists, but are immutable.  Tuples are created by separating objects with commas.

\lstset{language=Python,label= ,caption= ,captionpos=b,numbers=none}
\begin{lstlisting}
>>> pair = 1, 2
>>> pair
(1, 2)
\end{lstlisting}

Tuples can be used in assignments to "unpack" a sequence

\lstset{language=Python,label= ,caption= ,captionpos=b,numbers=none}
\begin{lstlisting}
>>> a, b = [1, 2]
>>> a
1
>>> b
2
\end{lstlisting}

Tuple assignment can be used to swap values

\lstset{language=Python,label= ,caption= ,captionpos=b,numbers=none}
\begin{lstlisting}
>>> b, a = a, b
>>> a, b
(2, 1)
\end{lstlisting}
\end{frame}


\begin{frame}[label={sec:orgf97d646}]{Conclusion}
Typical Python programs make extensive use of built-in data structures and often combine them (lists of lists, dictionaries of lists, etc)

\begin{itemize}
\item These are just the basics
\item Explore these data structures on your own
\item Read the books and Python documentation
\end{itemize}


This is a small taste of the expressive power and syntactic convenience of Python's data structures.
\end{frame}
\end{document}