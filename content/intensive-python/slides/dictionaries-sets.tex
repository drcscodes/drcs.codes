% Created 2022-08-15 Mon 12:56
% Intended LaTeX compiler: pdflatex
\documentclass[smaller, aspectratio=1610]{beamer}
\usepackage[utf8]{inputenc}
\usepackage[T1]{fontenc}
\usepackage{graphicx}
\usepackage{longtable}
\usepackage{wrapfig}
\usepackage{rotating}
\usepackage[normalem]{ulem}
\usepackage{amsmath}
\usepackage{amssymb}
\usepackage{capt-of}
\usepackage{hyperref}
\setbeamertemplate{navigation symbols}{}
\usepackage{verbatim, multicol, tabularx}
\usepackage{sourcecodepro}
\usepackage[T1]{fontenc}
\usepackage{amsmath,amsthm, amssymb, latexsym, listings, qtree}
\lstset{extendedchars=\true, inputencoding=utf8, frame=tb, aboveskip=1mm, belowskip=0mm, showstringspaces=false, columns=fixed, basicstyle={\footnotesize\ttfamily}, numbers=left, frame=single, breaklines=true, breakatwhitespace=true, tabsize=4,  keywordstyle=\color{blue}, identifierstyle=\color{violet}, stringstyle=\color{teal}, commentstyle=\color{darkgray}}
\setbeamertemplate{footline}[frame number]
\hypersetup{colorlinks=true,urlcolor=blue,bookmarks=true}
\setlength{\parskip}{.25\baselineskip}
\usetheme{default}
\date{}
\title{Dictionaries and Sets}
\hypersetup{
 pdfauthor={},
 pdftitle={Dictionaries and Sets},
 pdfkeywords={},
 pdfsubject={},
 pdfcreator={Emacs 28.1 (Org mode 9.5.2)},
 pdflang={English}}
\begin{document}

\maketitle

\section{Dictionaries and Sets}
\label{sec:org2476c9a}

\begin{frame}[label={sec:orgd65daaa}]{Dictionaries and Sets}
\begin{itemize}
\item Dictionaries map keys to values
\item Sets represent mathematical sets
\end{itemize}

Byt the end of this lesson you will

\begin{itemize}
\item know how to use dictionaries
\item know jow to use sets
\end{itemize}
\end{frame}

\begin{frame}[label={sec:orgc3f7401},fragile]{Dictionaries}
 A dictionary is a map from keys to values.

Create dictionaries with \texttt{\{\}}

\lstset{language=Python,label= ,caption= ,captionpos=b,numbers=none}
\begin{lstlisting}
>>> capitals = {}
\end{lstlisting}

Add key-value pairs with assignment operator

\lstset{language=Python,label= ,caption= ,captionpos=b,numbers=none}
\begin{lstlisting}
>>> capitals['Georgia'] = 'Atlanta'
>>> capitals['Alabama'] = 'Montgomery'
>>> capitals
{'Georgia': 'Altanta', 'Alabama': 'Montgomery'}
\end{lstlisting}

Keys are unique, so assignment to same key updates mapping

\lstset{language=Python,label= ,caption= ,captionpos=b,numbers=none}
\begin{lstlisting}
>>> capitals['Alabama'] = 'Birmingham'
>>> capitals
{'Georgia': 'Altanta', 'Alabama': 'Birmingham'}
\end{lstlisting}
\end{frame}

\begin{frame}[label={sec:org5955066},fragile]{Dictionary Operations}
 Remove a key-value mapping with \texttt{del} statement

\lstset{language=Python,label= ,caption= ,captionpos=b,numbers=none}
\begin{lstlisting}
>>> del capitals['Alabama']
>>> capitals
{'Georgia': 'Atlanta'}
\end{lstlisting}

Use the \texttt{in} operator to test for existence of key (not value)

\lstset{language=Python,label= ,caption= ,captionpos=b,numbers=none}
\begin{lstlisting}
>>> 'Georgia' in capitals
True
>>> 'Atlanta' in capitals
False
\end{lstlisting}

Extend a dictionary with \texttt{update()} method, get values as a list
with values method

\lstset{language=Python,label= ,caption= ,captionpos=b,numbers=none}
\begin{lstlisting}
>>> capitals.update({'Tennessee': 'Nashville', 'Mississippi':
'Jackson'})
>>> capitals.values()
dict_values(['Jackson', 'Nashville', 'Atlanta'])
\end{lstlisting}
\end{frame}

\begin{frame}[label={sec:orgb8ee536},fragile]{Conversions to \texttt{dict}}
 Any sequence of two-element sequences can be converted to a \texttt{dict}

A list of two-element lists:

\lstset{language=Python,label= ,caption= ,captionpos=b,numbers=none}
\begin{lstlisting}
>>> dict([[1, 1], [2, 4], [3, 9], [4, 16]])
{1: 1, 2: 4, 3: 9, 4: 16}
\end{lstlisting}

A list of two-element tuples:


\lstset{language=Python,label= ,caption= ,captionpos=b,numbers=none}
\begin{lstlisting}
>>> dict([('Lassie', 'Collie'), ('Rin Tin Tin', 'German
Shepherd')])
{'Rin Tin Tin': 'German Shepherd', 'Lassie': 'Collie'}
\end{lstlisting}

Even a list of two-character strings:

\lstset{language=Python,label= ,caption= ,captionpos=b,numbers=none}
\begin{lstlisting}
>>> dict(['a1', 'a2', 'b3', 'b4'])
{'b': '4', 'a': '2'}
\end{lstlisting}

Notice that subsequent pairs overwrote previously set keys.
\end{frame}

\begin{frame}[label={sec:org3b44587},fragile]{Sets}
 Sets have no duplicates, like the keys of a \texttt{dict}. They can be iterated
over (we'll learn that later) but can't be accessed by index.

\begin{itemize}
\item Create an empty set with \texttt{set()} function, add elements with \texttt{add()} method
\end{itemize}

\lstset{language=Python,label= ,caption= ,captionpos=b,numbers=none}
\begin{lstlisting}
>>> names = set()
>>> names.add('Ally')
>>> names.add('Sally')
>>> names.add('Mally')
>>> names.add('Ally')
>>> names
{'Ally', 'Mally', 'Sally'}
\end{lstlisting}

\begin{itemize}
\item Converting to set a convenient way to remove duplicates
\end{itemize}

\lstset{language=Python,label= ,caption= ,captionpos=b,numbers=none}
\begin{lstlisting}
>>> set([1,2,3,4,3,2,1])
{1, 2, 3, 4}
\end{lstlisting}
\end{frame}

\begin{frame}[label={sec:orgda4821b},fragile]{Set Operations}
 Intersection (elements in \texttt{a} \alert{and} \texttt{b})

\lstset{language=Python,label= ,caption= ,captionpos=b,numbers=none}
\begin{lstlisting}
>>> a = {1, 2}
>>> b = {2, 3}
>>> a & b # or a.intersetion(b)
{2}
\end{lstlisting}

Union (elements in \texttt{a} \alert{or} \texttt{b})

\lstset{language=Python,label= ,caption= ,captionpos=b,numbers=none}
\begin{lstlisting}
>>> a | b # or a.union(b)
{1, 2, 3}
\end{lstlisting}
\end{frame}

\begin{frame}[label={sec:org75ce0ae},fragile]{Set Operations}
 Difference (elements in \texttt{a} that are not in \texttt{b})

\lstset{language=Python,label= ,caption= ,captionpos=b,numbers=none}
\begin{lstlisting}
>>> a - b # or a.difference(b)
{1}
\end{lstlisting}

Symmetric difference (elements in \texttt{a} or \texttt{b} but not both)

\lstset{language=Python,label= ,caption= ,captionpos=b,numbers=none}
\begin{lstlisting}
>>> a ^ b # or a.symmetric_difference(b)
{1, 3}
\end{lstlisting}
\end{frame}

\begin{frame}[label={sec:org7c424ca},fragile]{Set Predicates}
 A predicate function asks a question with a \texttt{True} or \texttt{False} answer.

Subset of:

\lstset{language=Python,label= ,caption= ,captionpos=b,numbers=none}
\begin{lstlisting}
>>>a <= b # or a.issubset(b)
False
\end{lstlisting}

Proper subset of:

\lstset{language=Python,label= ,caption= ,captionpos=b,numbers=none}
\begin{lstlisting}
>>> a < b
False
\end{lstlisting}
\end{frame}

\begin{frame}[label={sec:org5100bd4},fragile]{Set Predicates}
 Superset of:

\lstset{language=Python,label= ,caption= ,captionpos=b,numbers=none}
\begin{lstlisting}
>>> a >= b # or a.issuperset(b)
False
\end{lstlisting}

Proper superset of:

\lstset{language=Python,label= ,caption= ,captionpos=b,numbers=none}
\begin{lstlisting}
>>> a > b
False
\end{lstlisting}
\end{frame}

\begin{frame}[label={sec:orgb5b1ee2}]{Closing Thoughts}
Typical Python programs make extensive use of built-in data structures and often combine them (lists of lists, dictionaries of lists, etc)

\begin{itemize}
\item These are just the basics
\item Explore these data structures on your own
\item Read the books and Python documentation
\end{itemize}


This is a small taste of the expressive power and syntactic
convenience of Python's data structures.
\end{frame}
\end{document}