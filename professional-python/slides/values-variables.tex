% Created 2022-08-23 Tue 17:44
% Intended LaTeX compiler: pdflatex
\documentclass[smaller, aspectratio=1610]{beamer}
\usepackage[utf8]{inputenc}
\usepackage[T1]{fontenc}
\usepackage{graphicx}
\usepackage{longtable}
\usepackage{wrapfig}
\usepackage{rotating}
\usepackage[normalem]{ulem}
\usepackage{amsmath}
\usepackage{amssymb}
\usepackage{capt-of}
\usepackage{hyperref}
\setbeamertemplate{navigation symbols}{}
\usepackage{verbatim, multicol, tabularx}
\usepackage{sourcecodepro}
\usepackage[T1]{fontenc}
\usepackage{amsmath,amsthm, amssymb, latexsym, listings, qtree}
\lstset{extendedchars=\true, inputencoding=utf8, frame=tb, aboveskip=1mm, belowskip=0mm, showstringspaces=false, columns=fixed, basicstyle={\footnotesize\ttfamily}, numbers=left, frame=single, breaklines=true, breakatwhitespace=true, tabsize=4,  keywordstyle=\color{blue}, identifierstyle=\color{violet}, stringstyle=\color{teal}, commentstyle=\color{darkgray}, upquote=false, literate={'}{\textquotesingle}1}
\setbeamertemplate{footline}[frame number]
\hypersetup{colorlinks=true,urlcolor=blue,bookmarks=true}
\setlength{\parskip}{.25\baselineskip}
\usetheme{default}
\date{}
\title{Values and Variables}
\hypersetup{
 pdfauthor={},
 pdftitle={Values and Variables},
 pdfkeywords={},
 pdfsubject={},
 pdfcreator={Emacs 28.1 (Org mode 9.5.2)},
 pdflang={English}}
\begin{document}

\maketitle


\section{Values and Variables}
\label{sec:org4a8e4a6}

\begin{frame}[label={sec:org63e45ad}]{Languages and Computation}
Every powerful language has three mechanisms for combining simple ideas to form more complex ideas:(\href{http://mitpress.mit.edu/sicp/full-text/book/book-Z-H-10.html}{SICP 1.1})

\begin{itemize}
\item \alert{primitive expressions}, which represent the simplest entities the language is concerned with,
\item \alert{means of combination}, by which compound elements are built from simpler ones, and
\item \alert{means of abstraction}, by which compound elements can be named and manipulated as units.
\end{itemize}

By the end of this lession you will
\begin{itemize}
\item know what a value is and how to create one,
\item know what an expression is how to combine them produce new values,
\item know what a type is and how it constrains what you can do with expressions, and
\item know what a variable is and how to use them as simple means of abstraction.
\end{itemize}
\end{frame}

\begin{frame}[label={sec:org62068d5}]{Values}
\begin{center}
\includegraphics[height=.7\textheight]{./value-uga-shirt.jpeg}
\end{center}
\end{frame}

\begin{frame}[label={sec:org67faa56},fragile]{Expressions}
 \begin{description}
\item[{value}] a well-defined chunk of data in memory
\item[{expression}] a sequence of symbols that can be \alert{evaluated} to produce a value
\end{description}

When you an expression into the Python REPL, Python evaluates it and prints its value.

\lstset{language=Python,label= ,caption= ,captionpos=b,numbers=none}
\begin{lstlisting}
>>> 1
1
>>> 3.14
3.14
>>> "pie"
'pie'
\end{lstlisting}

The simplest expressions are \alert{literal} values, as in the examples above.

\begin{description}
\item[{literal}] the textual representation of a value in source code.
\end{description}

Compound expressions combine values using operators.  Here the \texttt{+} operator combines the two literal values \texttt{2} and \texttt{3} -- the \alert{operands} -- to produce the value \texttt{5}:

\lstset{language=Python,label= ,caption= ,captionpos=b,numbers=none}
\begin{lstlisting}
>>> 2 + 3
5
\end{lstlisting}

Have a Python REPL session open for this lesson so you can follow along and try your own ideas.
\end{frame}

\begin{frame}[label={sec:org7fc6757},fragile]{Types}
 You can think of a type
\begin{itemize}
\item structurally: as an interpretation of the bits comprising a chunk of data,
\item denotationally: as a set of values, or
\item abstraction-based: as the set of operations available for a type.
\end{itemize}

All values have types. Python can tell you the type of a value with the built-in \texttt{type} function:

\lstset{language=Python,label= ,caption= ,captionpos=b,numbers=none}
\begin{lstlisting}
>>> type(1)
<class 'int'>
>>> type(3.14)
<class 'float'>
>>> type("pie")
<class 'str'>
\end{lstlisting}

\begin{block}{Active Review}
\begin{itemize}
\item What's the type of \texttt{'1'}?
\end{itemize}
\end{block}
\end{frame}

\begin{frame}[label={sec:orgc7c8d8f},fragile]{Variables}
 Think of variable as a name for a value. You bind a value to a variable using an assignment statement (or by passing an argument to a function), after which the variable \alert{denotes} the value:

\lstset{language=Python,label= ,caption= ,captionpos=b,numbers=none}
\begin{lstlisting}
>>> a = "Ok"
>>> a
'Ok'
\end{lstlisting}

\texttt{=} is the assignment operator.  An assignment statement has the form:

\begin{quote}
\texttt{<variable\_name> = <expression>}
\end{quote}

You can unbind a variable with the \texttt{del} function.

\lstset{language=Python,label= ,caption= ,captionpos=b,numbers=none}
\begin{lstlisting}
>>> del(a)
>>> a
Traceback (most recent call last):
  File "<stdin>", line 1, in <module>
NameError: name 'a' is not defined
\end{lstlisting}

Variable names, or identifiers, may contain letters, numbers, or underscores and may not begin with a number.

\begin{block}{Active Review}
\begin{itemize}
\item What happens when you execute this assignment statement?

\lstset{language=Python,label= ,caption= ,captionpos=b,numbers=none}
\begin{lstlisting}
  >>> 16_candles = "Molly Ringwald"
\end{lstlisting}
\end{itemize}
\end{block}
\end{frame}

\begin{frame}[label={sec:orgd32c467},fragile]{Keywords}
 Python reserves keywords for its own use.

\lstset{language=Python,label= ,caption= ,captionpos=b,numbers=none}
\begin{lstlisting}
>>> from keyword import kwlist
>>> import math
>>> numrows = 5
>>> numcols = math.ceil(len(kwlist) / numrows)
>>> for row in range(numrows):
...     for col in range(0, numrows * numcols, numrows):
...         kw = kwlist[row+col] if row+col < len(kwlist) else ''
...         print(f'{kw:<12}', end='')
...     print()
...
False       assert      continue    except      if          nonlocal    return
None        async       def         finally     import      not         try
True        await       del         for         in          or          while
and         break       elif        from        is          pass        with
as          class       else        global      lambda      raise       yield
\end{lstlisting}


\begin{block}{Active Review}
\begin{itemize}
\item What happens when you execute this assignment statement?

\lstset{language=Python,label= ,caption= ,captionpos=b,numbers=none}
\begin{lstlisting}
  >>> class = "CS 2316"
\end{lstlisting}

\item What happens if you use \texttt{print} as a variable name?
\item How can you fix it?
\end{itemize}
\end{block}
\end{frame}

\begin{frame}[label={sec:org7b08dd1},fragile]{Statements vs Expressions}
 Expressions have values, statements only have effects -- like binding a variable to a value or or effecting control flow in a program. Assignment using \texttt{=} is a statement -- it cannot be used where a value is expected.

\lstset{language=Python,label= ,caption= ,captionpos=b,numbers=none}
\begin{lstlisting}
>>> while guess = input("Guess a number: "):
...     if guess == "7":
...         break
  Input In [42]
    while guess = input("Guess a number: "):
          ^
SyntaxError: invalid syntax. Maybe you meant '==' or ':=' instead of '='?
\end{lstlisting}

Since version 3.8, Python provides the "walrus" operator, \texttt{:=}, which creates an assignment expression, where the assigned value is the value of the expression:

\lstset{language=Python,label= ,caption= ,captionpos=b,numbers=none}
\begin{lstlisting}
>>> while guess := input("Guess a number: "):
...     if guess == "7":
...         break
...
Guess a number: 1
Guess a number: 7
>>>
\end{lstlisting}

Note that \texttt{while} and \texttt{if} are statements -- they don't produce values, they create effects.

We will see a few cases in future lessons where the walrus operator is helpful.
\end{frame}

\begin{frame}[label={sec:orgcd21d82},fragile]{Aside: The Sizes of Types}
 One of the convenient things about Python is that you don't have to worry about overflow or underflow\footnote{In regular Python you don't have to worry about type size limits, but in scientific Python, which relies on libraries written in C, C++ and Fortran you do.}. For example, as in mathematics, the set \texttt{int} is unbounded:

\lstset{language=Python,label= ,caption= ,captionpos=b,numbers=none}
\begin{lstlisting}
>>> import sys
>>> x = sys.maxsize
>>> x
9223372036854775807 # That's ~ 9.2 quintillion, i.e., 9.2e+18
>>> x = x + 1
>>> x
9223372036854775808
>>>
\end{lstlisting}

But you should consider \texttt{sys.maxsize}, the word size of your processor (64 bits in this example, since \texttt{sys.maxsize} \(= 2^{63} - 1\)), to be the practical limit, because it's the theoretical limit \footnote{Not strictly true, but practically true.} of addressable RAM and thus the largest possible (but certainly impractical) array you could store in main memory and therefore, as you'll learn later, the largest possible list index.

In many other programming languages, size limits can crop up in sometimes amusing ways, \href{https://arstechnica.com/information-technology/2014/12/gangnam-style-overflows-int\_max-forces-youtube-to-go-64-bit/}{Gangnam Style!}
\end{frame}

\begin{frame}[label={sec:org0e4322d},fragile]{Types as Sets of Operations}
 Types determine which operations are available on values. For example, exponentiation is defined for numbers (like int or float):

\lstset{language=Python,label= ,caption= ,captionpos=b,numbers=none}
\begin{lstlisting}
>>> 2**3
8
\end{lstlisting}

\ldots{} but not for \texttt{str} (string) values:

\lstset{language=Python,label= ,caption= ,captionpos=b,numbers=none}
\begin{lstlisting}
>>> "pie"**3
Traceback (most recent call last):
  File "<stdin>", line 1, in <module>
TypeError: unsupported operand type(s) for ** or pow(): 'str' and 'int'
\end{lstlisting}

This is the primary way to think about types in Python.
\end{frame}

\begin{frame}[label={sec:orga565fc5},fragile]{Overloaded Operators}
 Some operators are overloaded, meaning they have different meanings when applied to different types. For example, + means addition for numbers and concatenation for strings:

\lstset{language=Python,label= ,caption= ,captionpos=b,numbers=none}
\begin{lstlisting}
>>> 2 + 2
4
>>> "Yo" + "lo!"
'Yolo!'
\end{lstlisting}

\texttt{*} means multiplication for numbers and repetition for strings:

\lstset{language=Python,label= ,caption= ,captionpos=b,numbers=none}
\begin{lstlisting}
>>> 2 * 3
6
>>> "Yo" * 3
'YoYoYo'
>>> 3 * "Yo"
'YoYoYo'
\end{lstlisting}
\end{frame}

\begin{frame}[label={sec:org1ca505f},fragile]{Expression Evaluation}
 Mathematical expressions are evaluated using precedence and associativity rules as you would expect from math:

\lstset{language=Python,label= ,caption= ,captionpos=b,numbers=none}
\begin{lstlisting}
>>> 2 + 4 * 10
42
\end{lstlisting}

If you want a different order of operations, use parentheses:

\lstset{language=Python,label= ,caption= ,captionpos=b,numbers=none}
\begin{lstlisting}
>>> (2 + 4) * 10
60

\end{lstlisting}

Note that precedence and associativity rules apply to overloaded versions of operators as well:

\lstset{language=Python,label= ,caption= ,captionpos=b,numbers=none}
\begin{lstlisting}
>>> "Honey" + "Boo" * 2
'HoneyBooBoo'
\end{lstlisting}

\begin{block}{Active Review}
\begin{itemize}
\item How could we modify the expression above to evaluate to 'HoneyBooHoneyBoo' ?
\end{itemize}
\end{block}
\end{frame}

\begin{frame}[label={sec:org836ff06},fragile]{Python is Dynamically Typed}
 Python is dynamically typed, meaning that types are not resoved until run-time. This means two things practically:

\begin{enumerate}
\item Values have types, variables don't:
\lstset{language=Python,label= ,caption= ,captionpos=b,numbers=none}
\begin{lstlisting}
   >> a = 1
   >>> type(a)
   <class 'int'>
   >>> a = 1.1 # would be disallowed in a statically typed language
   >>> type(a)
   <class 'float'>
\end{lstlisting}
\item Python doesn't report type errors until run-time. We'll see many examples of this fact.
\end{enumerate}

\begin{block}{Active Review}
Evaluate the following expressions in the Python REPL.  Be sure to type them exactly as written.

\begin{itemize}
\item \texttt{2 + 3}
\item \texttt{"2" + "3"}
\item \texttt{"2" + 3}
\item \texttt{2 + "3"}
\end{itemize}
\end{block}
\end{frame}

\begin{frame}[label={sec:org555a2d7},fragile]{Type Conversions}
 Convert a value to a different type by applying conversions named after the target type.

\lstset{language=Python,label= ,caption= ,captionpos=b,numbers=none}
\begin{lstlisting}
>>> int(2.9)
2
>>> float(True)
1.0
>>> int(False)
0
>>> str(True)
'True'
>>> int("False")
Traceback (most recent call last):
  File "<stdin>", line 1, in <module>
ValueError: invalid literal for int() with base 10: 'False'
\end{lstlisting}

\begin{block}{Active Review}
Modify the following expressions to produce the indicated results.

\begin{itemize}
\item \texttt{"2" + 3} (we want \texttt{"23"})
\item \texttt{2 + "3"} (we want \texttt{5})
\end{itemize}
\end{block}
\end{frame}

\begin{frame}[label={sec:org06a0fba},fragile]{Boolean Values}
 There are \texttt{10} kinds of people:

\begin{itemize}
\item those who know binary, and
\item those who don't.
\end{itemize}
\end{frame}

\begin{frame}[label={sec:org457d145},fragile]{Python Booleans}
 In Python, boolean values have the \texttt{bool} type. Four kinds of boolean
expressions:

\begin{itemize}
\item \texttt{bool} literals: \texttt{True} and \texttt{False}
\item \texttt{bool} variables
\item expressions formed by combining non-\texttt{bool} expressions with comparison operators
\item expressions formed by combining \texttt{bool} expressions with logical operators
\end{itemize}
\end{frame}

\begin{frame}[label={sec:org1a2935d},fragile]{Comparison Operators}
 \begin{itemize}
\item Equal to: \texttt{==}, like \(=\) in math

\begin{itemize}
\item Remember, \texttt{=} is assignment operator, \texttt{==} is comparison operator!
\end{itemize}

\item Not equal to: \texttt{!=}, like \(\ne\) in math
\item Greater than: \texttt{>}, like \(>\) in math
\item Greater than or equal to: \texttt{>=}, like \(\ge\) in math
\end{itemize}

\lstset{language=Python,label= ,caption= ,captionpos=b,numbers=none}
\begin{lstlisting}
1 == 1 # True
1 != 1 # False
1 >= 1 # True
1 > 1  # False
\end{lstlisting}

\begin{block}{Active Review}
\begin{itemize}
\item What is the value of \texttt{"foo" == "Foo"}?
\item What is the value of \texttt{"foo" > "Foo"}?
\end{itemize}
\end{block}
\end{frame}

\begin{frame}[label={sec:org621c3c0},fragile]{Logical Operators}
 The values produced by logial operators are often shown in truth tables:

\begin{center}
\begin{tabular}{|l|l|l|l|l|}
\hline
a & b & not a & a and b & a or b \\
\hline
False & False & True & False & False \\
False & True & True & False & True \\
True & False & False & False & True \\
True & True & False & True & True \\
\hline
\end{tabular}
\end{center}

Some equivalent Python expressions:

\lstset{language=Python,label= ,caption= ,captionpos=b,numbers=none}
\begin{lstlisting}
True and True  # True
True and False # False
True or False  # True
False or False # False
not True       # False
\end{lstlisting}
\end{frame}

\begin{frame}[label={sec:orga74eeaa},fragile]{Truth in Python}
 The zero values of built-in types are equivalent to \texttt{False}:

\begin{itemize}
\item boolean \texttt{False}
\item \texttt{None}
\item integer \texttt{0}
\item float \texttt{0.0}
\item empty string \texttt{""}
\item empty list \texttt{[]}
\item empty tuple \texttt{()}
\item empty dict \texttt{\{\}}
\item empty set \texttt{set()}
\end{itemize}

All other values are equivalent to True.

\begin{itemize}
\item Every value in Python is either \alert{truthy} or \alert{falsey} and can be used in a boolean context.
\end{itemize}
\end{frame}

\begin{frame}[label={sec:org8794f38},fragile]{Short-circuit Evaluation}
 Logical expressions use short-circuit evaluation:

\begin{itemize}
\item \texttt{or} only evaluates second operand if first operand is \texttt{False}
\item \texttt{and} only evaluates second operand if first operand is \texttt{True}
\end{itemize}

Guard idiom: \texttt{(b == 0) or print(a / b)}, or \texttt{(b != 0) and print(a / b)}

\begin{block}{Active Review}
What are the values of the following expressions?

\begin{itemize}
\item \texttt{True and False}
\item \texttt{True and 0}
\item \texttt{True and []}
\item \texttt{True and None}
\item \texttt{type(True and None)}
\item \texttt{False or 1}
\item \texttt{True or 1}
\item \texttt{1 and "done"}
\item \texttt{1 == 1 or 0}
\item \texttt{1 == 1 and 0}
\item \texttt{1 == (1 and 0)}
\end{itemize}
\end{block}
\end{frame}


\begin{frame}[label={sec:org182c96a},fragile]{Sequences}
 Sequences are ordered collections of objects -- lists, tuples, and strings.

\lstset{language=Python,label= ,caption= ,captionpos=b,numbers=none}
\begin{lstlisting}
>>> boys = ['Stan', 'Kyle', 'Cartman', 'Kenny']
>>> boys[0]
'Stan'
>>> empty = []
>>> also_empty = list()
\end{lstlisting}

Lists are mutable.

\lstset{language=Python,label= ,caption= ,captionpos=b,numbers=none}
\begin{lstlisting}
>>> boys[2] = 'Eric'
>>> boys
['Stan', 'Kyle', 'Eric', 'Kenny']
\end{lstlisting}

Tuples and strings are immutable.

\lstset{language=Python,label= ,caption= ,captionpos=b,numbers=none}
\begin{lstlisting}
>>> pair = 1, 2
>>> pair
(1, 2)
>>> pair[0] = 0
Traceback (most recent call last):
  Input In [37] in <cell line: 1>
    pair[0] = 0
TypeError: 'tuple' object does not support item assignment
\end{lstlisting}
\end{frame}


\begin{frame}[label={sec:org2bf3e69},fragile]{Dictionaries and Sets}
 A dictionary is a map from keys to values.

\lstset{language=Python,label= ,caption= ,captionpos=b,numbers=none}
\begin{lstlisting}
>>> capitals = {}
>>> capitals['Georgia'] = 'Atlanta'
>>> capitals['Alabama'] = 'Montgomery'
>>> capitals
{'Georgia': 'Altanta', 'Alabama': 'Montgomery'}
>>> capitals['Georgia']
'Atlanta'
\end{lstlisting}

Sets have no duplicates, like the keys of a \texttt{dict}. They can be iterated over (we'll learn that later) but can't be accessed by index.

\lstset{language=Python,label= ,caption= ,captionpos=b,numbers=none}
\begin{lstlisting}
>>> names = set()
>>> names.add('Ally')
>>> names.add('Sally')
>>> names.add('Mally')
>>> names.add('Ally')
>>> names
{'Ally', 'Mally', 'Sally'}
>>> set([1,2,3,4,3,2,1])  # Removes duplicates
{1, 2, 3, 4}
\end{lstlisting}
\end{frame}


\begin{frame}[label={sec:orgaf2afe3}]{Values, Variables, and Expressions}
\begin{itemize}
\item Values are the atoms of computer programs
\item Expressions produce values
\item We combine values using operators and functions to form compound expressions
\item Variables are identifiers that denote values
\begin{itemize}
\item Identifiers also denote functions, classes, modules and packages
\end{itemize}
\item Choose identifiers carefully to create beautiful, readable programs
\end{itemize}
\end{frame}
\end{document}